\subsection{Hash functions}
Hash functions are a fundamental tool in many fields of computer science, and cryptography is
arguably the most prominent.
Formally, an hash function is any function \(H\colon {\{0, 1\}}^{*} \mapsto {\{0, 1\}}^n\), that
is any function which maps arbitrarly long \emph{messages} to fixed-size \emph{digests}.
From the definition, it is immediate to see that there are an infinite number of messages which map
to the same digest.
While an operation like truncation is a (very simple) hash function, in cryptography we are
interested in functions that provide additional guarantees: the assumption is that a digest sohuld
represent a message in a one-way fashion: while there are infinite messages which map to the same
digest, it must be hard to find them.
Ideally, a cryptographic hash function should behave like a perfect random function.
This is of course impossible, as the output of an hash function must only depend deterministically
on its input; the aim then is to build functions which are hard to distinguish from a random
function.
\begin{definition}[Cryptographic hash function]
	Given \(n \in \mathbb{N}\), an \emph{\(n\)(-bit) cryptographic hash function (CHF)} is any
	function \(H\colon {\{0, 1\}}^{*} \mapsto {\{0, 1\}}^n\) which satisfies the following properties:
	\begin{itemize}
		\item \textbf{Collision resistance}: It is hard to find two messages \(m_1, m_2\) such
		      that \(H(m_1) = H(m_2)\).
		\item \textbf{Preimage resistance}: Given some digest \(h\), it is hard to find a
		      message \(m\) such that \(H(m) = h\) (\(H\) is a one-way function).
		\item \textbf{Second preimage resistance}: Given some message \(m_1\), it is hard to
		      find a message \(m_2\) such that \(H(m_1) = H(m_2)\).
	\end{itemize}
\end{definition}

\noindent While some of the requirements might seem redundant (for example, if it is hard for an
attacker to find a collision for chosen messages, it must be hard when one is fixed),
the difference usually lies in how exactly we define hardness for each property.
For collision resistance, an ideal CHF requires about \(2^{n/2}\) evaluations to find a collision
(birthday paradox), while for preimage resistance it would require about \(2^n\) evaluations.
Tipically, a CHF is built by applying some known secure constructions to functions which are
simpler to devise.
\begin{definition}[Pseudorandom keyed permutation]
	Given \(l, n \in \mathbb{N}\), an \emph{\(l/n\)(-bit) pseudorandom keyed permutation (PKP)} is
	any bijective function:
	\[F\colon {\{0, 1\}}^l \times {\{0, 1\}}^n \mapsto {\{0, 1\}}^l\]
	which is hard to distinguish from an uniform random distribution.
\end{definition}

\noindent PKPs are often built by iterating a keyed permutation \(F\) for some number \(r\) of
rounds, since \(F\) by itself might be relatively easy to invert.
A block cipher is a pseudorandom keyed permutation which changes the key being used in each
round through a key-scheduling function.
Unkeyed permutations can be derived from keyed ones simply by fixing the key to some arbitrary
value.
\begin{definition}[One-way compression function]
	Given \(l_1, l_2, n \in \mathbb{N}\), an \emph{\(l_1/n/l_2\)(-bit) one-way compression function
		(OWCF)}
	is any function:
	\[F\colon {\{0, 1\}}^{l_1} \times {\{0, 1\}}^n \mapsto {\{0, 1\}}^{l_2}\]
\end{definition}

\noindent There are many known ways to build OWCFs from pseudorandom keyed
permutations, and, in turn, CHFs from OWCFs.
We will introduce the Davies-Meyer and the Merkle-Damg\r{a}rd constructions respectively, as 
those are the ones of interest to us.
\begin{theorem}[Davies-Meyer construction]
	Given a \(l/n\) pseudorandom keyed permutation \(E\), some \(i, k \in \mathbb{N}\), some
	\(v \in {\{0, 1\}}^l\), and some \(m \in {\{0, 1\}}^{kn}\), then any function \(F_E\) such that:
	\begin{align*}
		 & F_{E,i}(v, m) =
		\begin{cases}
			E(v, m_{\range{1}{n}})                      & i = 1         \\
			E(F_{E, i-1}(v, m), m_{\range{i(n-1)}{in}}) & 2 \le i \le k \\
		\end{cases} \\
		 & F_E = F_{E, k}
	\end{align*}
	is a \(l/kn/l\) OWCF\@.
\end{theorem}
\begin{theorem}[Merkle-Damg\r{a}rd construction]
	Given a \(l_1/n/l_2\) OWCF \(F\), some \(k \in \mathbb{N}\), some \(v \in {\{0, 1\}}^{l_1}\),
	some \(m \in {\{0, 1\}}^*\) and some padding funcion:
	\[P(m)\colon {\{0, 1\}}^{|m|} \mapsto {\{0, 1\}}^{|m| + (-|m| \bmod n) + kn}\]
	such that, \(\forall m, m' \in {\{0, 1\}}^*\colon \)
	\[(|m| = |m'| \then |P(m)| = |P(m')|) \land (|m| \neq |m'| \then m_{|P(m)|} \neq m'_{|P(m')|})\]
	then any function \(H_F\) such that:
	\begin{align*}
		 & H_{F, i}(v, m) =
		\begin{cases}
			F(v, m_1)                & i = 1            \\
			F(H_{F, i-1}(v, m), m_i) & 1 < i \le |P(m)| \\
		\end{cases} \\
		 & H_F = H_{F, |P(m)|}
	\end{align*}
	is a cryptographic hash function.
\end{theorem}
