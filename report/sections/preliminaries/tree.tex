\subsection{Tree hash modes}
An important application of CHFs is in \emph{prover-verifier games}:
for any message \(m\), the digest \(h = H\left(m\right)\), where \(H\) is an \(n\) CHF, can be
used as a \emph{binding commitment} for \(m\): a verifier is convinced that the prover knows \(m\)
simply by asking him to share \(h\), with overwhelming confidence (\(\approx 1 - \frac{1}{2^n}\)).
While often referred to as if they were humans, provers and verifiers are formally described by
some model of computation, usually deterministic Turing machines, which often can only harness a
limited amount of resources (time and space), tipically at most polynomial in the size of the game
instance statement\footnote{Although humans can be  assimilated to a computational model, it is
	not easy to formalize the eventuality of the prover threatening the verifier to make him accept
	his proof\dots}.

If the prover wants to commit to a list of \(k\) messages, a possibility would be to share with the
verifier the hash of every message: this would require a \(\BigO\left(k\right)\) communication cost
and a \(\BigO\left(k\right)\) verification cost.
A slightly better alternative would be for the prover to share
\(H\left(\left\{m_1, \dots, m_k\right\}\right)\): the communication cost would only be
\(\BigO\left(1\right)\), but verification would still cost \(\BigO\left(k\right)\).
\begin{definition}[Merkle Tree~\cite{Merkle1988}]
	Given some \(k \in \mathbb{N}\), a CHF \(H\) and a set of messages
	\(S = \left\{m_1, \dots, m_{s^{k-1}} \mid \forall i\colon m_i \in
	{\left\{0, 1\right\}}^*\right\} \), a \emph{Merkle Tree (MT)} is a complete binary tree of
	height \(k\) such that:
	\begin{enumerate}
		\item The leaf nodes \(\nu_1, \dots, \nu_{2^{k-1}}\) contain \(H\left(m_1\right), \dots,
		      H\left(m_{2^{k-1}}\right)\).
		\item Every other node \(\nu \) contains \(H\left(\nu_l, \nu_r\right)\), where \(\nu_l\) is
		      the left child of \(\nu \) and \(\nu_r\) is the right child of \(\nu \).
	\end{enumerate}
\end{definition}

\noindent By using Merkle trees, the prover only needs to send to the verifier, as a commitment for
some message \(m_i\) among \(k = 2^{\left\lfloor\log_2(k)\right\rfloor}\) messages, the contents of
the co-path from the leaf containing \(m_i\) to the root (plus the hash of \(m_i\)): this requires
just \(\BigO\left(\log_2\left(k\right)\right)\) communication effort and
\(\BigO\left(\log_2\left(k\right)\right)\) verification effort.
Another advantage of Merkle trees is that bottom-up construction is very easy to parallelize,
and its usefulness can be appreciated even more when considering a scenario where different
messages actually belong to different provers.
\begin{definition}[Augmented Binary tRee~\cite{AndreevaBR2021}]
	Given some \(k \in \mathbb{N}\), a CHF \(H\), and a set of messages
	\(S = \left\{m_1, \dots, m_{2^{k-1} + 2^{k-2}-1} \mid \forall i\colon m_i \in
	{\left\{0, 1\right\}}^*\right\} \),
	an \emph{Augmented Binary tRee (ABR)} is a complete binary tree of
	height \(k\) augmented with \emph{middle} nodes, such that:
	\begin{enumerate}
		\item The leaf nodes \(\nu_{1}, \dots, \nu_{2^{k-1}}\) contain \(H\left(m_1\right), \dots,
		      H\left(m_{2^{k-1}}\right)\).
		\item There are no middle nodes in the leaf layer.
		\item The middle nodes \(\nu_{2^{k-1}+1}, \dots, \nu_{\left|S\right|}\) contain
		      \(H\left(m_{2^{k-1}+1}\right), \dots, H\left(m_{\left|S\right|}\right)\).
		\item Every other node \(\nu \) contains \(H\left(\nu_l \oplus \nu_m, \nu_r \oplus
		      \nu_m\right) \oplus \nu_r \), where \(\nu_l\) is the left child of \(\nu \), \(\nu_r\)
		      is the right child of \(\nu \), and \(\nu_m\) is the middle child of \(\nu \), or \(0\)
		      if \(\nu \) doesen't have a middle child.
	\end{enumerate}
\end{definition}

\noindent Notice the use of the \(\oplus \) operation inside the ABR\@: while messages of length 
\(n\) are usually treated as elements of \({\left\{0, 1\right\}}^n\), they can also be treated as 
\(n\)-bit integers over some field \(\mathbb{F}_q\), so \(\oplus \) represents addition over the 
field of choice.

ABRs can store 50\% more messages than Merkle Trees for the same height, resulting in the same 
number of calls to \(H\), at the cost of performing 3 additional \(\oplus \) operations for
every call, whose cost is almost negligible, especially in ZK systems.
