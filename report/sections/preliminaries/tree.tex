\subsection{Tree hash modes}
An important application of CHFs is in \emph{prover-verifier games}:
for any message \(m\), the digest \(h = H(m)\), where \(H\) is an \(n\) CHF, can be
used as a \emph{binding commitment} for \(m\): a verifier is convinced that the prover knows \(m\)
simply by asking him to share \(h\), with overwhelming confidence (\(\approx 1 - \frac{1}{2^n}\)).
While often referred to as if they were humans, provers and verifiers are formally described by
some model of computation, usually deterministic Turing machines, which often can only harness a
limited amount of resources (time and space), tipically at most polynomial in the size of the game
instance statement\footnote{Although humans can be  assimilated to a computational model, it is
	not easy to formalize the eventuality of the prover threatening the verifier to make him accept
	his proof\dots}.

If the prover wants to commit to a list of \(k\) messages, a possibility would be to share with the
verifier the hash of every message: this would require a \(\BigO(k)\) communication cost and a
\(\BigO(k)\) verification cost.
A slightly better alternative would be for the prover to share \(H(\{m_1, \dots, m_k\})\): the
communication cost would only be \(\BigO(1)\), but verification would still cost \(\BigO(k)\).
\begin{definition}[Merkle Tree]
	Given some \(k \in \mathbb{N}\), a CHF \(H\) and a set of messages
	\(S = \{m_1, \dots, m_{s^{k-1}} \mid \forall i\colon m_i \in {\{0, 1\}}^*\} \),
	a \emph{Merkle Tree (MT)} is a complete binary tree of height \(k\) such that:
	\begin{enumerate}
		\item The leaf nodes \(\nu_1, \dots, \nu_{2^{k-1}}\) contain \(H(m_1), \dots,
		      H(m_{2^{k-1}})\).
		\item Every other node \(\nu \) contains \(H(\nu_l, \nu_r)\), where \(\nu_l\) is the left
		      child of \(\nu \) and \(\nu_r\) is the right child of \(\nu \).
	\end{enumerate}
\end{definition}

\noindent By using Merkle trees, the prover only needs to send to the verifier, as a commitment for
some message \(m_i\) among \(k = 2^{\lfloor\log_2(k)\rfloor}\) messages, the contents of the
co-path from the leaf containing \(m_i\) to the root (plus the hash of \(m_i\)): this requires just
\(\BigO(\log_2(k))\) communication effort and \(\BigO(\log_2(k))\) verification effort.
Another advantage of Merkle trees is that bottom-up construction is very easy to parallelize,
and its usefulness can be appreciated even more when considering a scenario where different
messages actually belong to different provers.
\begin{definition}[Augmented Binary tRee]
	Given some \(k \in \mathbb{N}\), a CHF \(H\), and a set of messages
	\(S = \{m_1, \dots, m_{2^{k-1} + 2^{k-2}-1} \mid \forall i\colon m_i \in {\{0, 1\}}^*\} \),
	an \emph{Augmented Binary tRee (ABR)} is a complete binary tree of
	height \(k\) augmented with \emph{middle} nodes, such that:
	\begin{enumerate}
		\item The leaf nodes \(\nu_{1}, \dots, \nu_{2^{k-1}}\) contain \(H(m_1), \dots,
		      H(m_{2^{k-1}})\).
		\item There are no middle nodes in the leaf layer.
		\item The middle nodes \(\nu_{2^{k-1}+1}, \dots, \nu_{|S|}\) contain
		      \(H(m_{2^{k-1}+1}), \dots, H(m_{|S|})\).
		\item Every other node \(\nu \) contains \(H(\nu_l \oplus \nu_m, \nu_r \oplus \nu_m)
		      \oplus \nu_r \), where \(\nu_l\) is the left child of \(\nu \), \(\nu_r\) is the right
		      child of \(\nu \), and \(\nu_m\) is the middle child of \(\nu \), or
		      \(0\) if \(\nu \) doesen't have a middle child.
	\end{enumerate}
\end{definition}

\noindent Notice the use of the \(\oplus \) operation inside the ABR\@: while messages of length \(n\)
are usually treated as elements of \({\{0, 1\}}^n\), they can also be treated as \(n\)-bit integers
over some field \(\mathbb{F}_q\): if \(q = 2^n\), then \(\oplus \) means bitwise \textsc{xor}
(i.e.\ addition in \(\mathbb{F}_2\)), and if \(q = p\) for some prime \(p\), then
\(\oplus \) means addition in the field \(\mathbb{F}_p\).

ABRs can store 50\% more messages than Merkle Trees for the same height, resulting in the
same number of calls to \(H\), at the cost of performing 3 additional \(\oplus \) operations for
every call (usually, \(TIME(\oplus) \ll TIME(H)\)).
