\subsection{Quadratic Arithmetic Programs}
A problem with R1CS is that solutions have size linear in the number of multiplication gates of
the corresponding arithmetic circuit.
This can be solved by using Quadratic Arithmetic Programs.
\begin{definition}[Quadratic Arithmetic Program]
	Given a field \(\mathbb{F}\) and some \(n,m \in \mathbb{N}\), a
	\emph{\(n/m\) Quadratic Arithmetic Program (QAP)} over \(\mathbb{F}\) is any quadruple:
	\[ \mathcal{Q} = \left(t, \bm{v}, \bm{w}, \bm{y}\right) \mid {t \in \mathbb{F}\left[x\right]}
		\land {\bm{v},\bm{w},\bm{y} \in {\mathbb{F}\left[x\right]}^n}\]
	For which it holds that:
	\[
		\forall i\colon \deg\left(\bm{v}_i\right)+1 = \deg\left(\bm{w}_i\right)+1 =
		\deg\left(\bm{y}_i\right)+1 = \deg\left(t\right) = m
	\]
	A \emph{valid assignment} to a QAP \(\mathcal{Q}\) is any vector:
	\[
		\bm{s} \in \mathbb{F}^n \mid \left(\bm{v}\bm{s}\right)\left(\bm{w}\bm{s}\right) -
		\bm{y}\bm{s} \bmod t = 0
	\]
	Then, the polynomials \(p = \left(\bm{v}\bm{s}\right)\left(\bm{w}\bm{s}\right) - \bm{y}\bm{s}\)
	and \(h = \frac{p}{t}\) are a \emph{solution} to \(\mathcal{Q}\).
\end{definition}

\noindent Just like it was possible to represent any \(n/m\) arithmetic circuit \(\phi \) with an
\(n/(n+m+1)\) R1CS \(\mathcal{C}\), we can, in turn, represent any \(n/m\) R1CS
\(\mathcal{C} = \left(\bm{A}, \bm{B}, \bm{C}\right)\) with a \(n/m\) QAP \(\mathcal{Q}\).
First, we choose some arbitrary
\(\bm{z} \in \mathbb{F}^n \mid \forall i,j\colon \bm{z}_i \neq \bm{z}_j\)
(usually, \(\bm{z} = \begin{pmatrix}1 & \cdots & n\end{pmatrix}\)).
Let \(\bm{Z} \in \mathbb{F}^{m \times n} \mid \forall i\colon \bm{Z}_i = \bm{z}\), then:
\[\mathcal{Q} = \left(t, \bm{v}, \bm{w}, \bm{y}\right) = \left(
	\prod_{i}{\left(x - \bm{z}_i\right)},
	L\left(\bm{Z}, \bm{A}^{\transpose}\right),
	L\left(\bm{Z}, \bm{B}^{\transpose}\right),
	L\left(\bm{Z}, \bm{C}^{\transpose}\right)
	\right)
\]
To make things more clear, let's make an example:
\begin{example}
	We want to compute the \(3/6\) QAP \(\mathcal{Q} = \left(t, \bm{v}, \bm{w}, \bm{y}\right)\)
	associated with the \(3/6\) R1CS \(\mathcal{C} = \left(\bm{A}, \bm{B}, \bm{C}\right)\) that we
	derived in Example~\ref{ex:r1cs}.
	First, we set:
	\begin{align*}
		 & \bm{z} = \begin{pmatrix}1 & 2 & 3\end{pmatrix} &  &
		\bm{Z} = \begin{pmatrix}\bm{z}; \bm{z}; \bm{z}; \bm{z}; \bm{z}; \bm{z}\end{pmatrix}
	\end{align*}
	Then, we compute the target polynomial \(t\), the left and right input constraint polynomial
	vectors \(\bm{v}\) and \(\bm{w}\), and the output constraint polynomial vector \(\bm{y}\)
	(remember, we are working over \(\mathbb{F}_{13}\)).
	Notice how the 2nd, 4th and 5th columns of \(\bm{A}\) form the canonical basis of
	\(\mathbb{F}_{13}^3\), and since \(L\) is a linear operator, we can express all other polynomials
	as linear combinations of \(L\left(\bm{z}, \bm{A}_2^{\transpose}\right), L\left(\bm{z},
	\bm{A}_4^{\transpose}\right)\) and \(L\left(\bm{z}, \bm{A}_5^{\transpose}\right)\):
	\begin{align*}
		 & t	 = \left(x - 1\right)\left(x - 2\right)\left(x - 3\right) =
		\left(x + 12\right)\left(x + 11\right)\left(x + 10\right) = x^3 + 7x^2 + 11x + 7 \\
		 & \bm{v} = L\left(\bm{Z}, \bm{A}^{\transpose}\right) =
		\begin{pmatrix}
			L\left(\bm{z}, \bm{A}^{\transpose}_{1}\right) & \cdots &
			L\left(\bm{z}, \bm{A}^{\transpose}_{6}\right)
		\end{pmatrix}
		= {\begin{pmatrix}
			   0               \\
			   7x^2 + 4x + 3   \\
			   0               \\
			   12x^2 + 4x + 10 \\
			   7x^2 + 5x + 1   \\
			   0
		   \end{pmatrix}}^\transpose                                             \\
		 & \bm{w} =
		\begin{pmatrix}
			0 & \bm{v}_2 + \bm{v_4} & \bm{v}_5 & 0 & 0 & 0
		\end{pmatrix}
		= {\begin{pmatrix}
			   0             \\
			   6x^2 + 8x     \\
			   7x^2 + 5x + 1 \\
			   0             \\
			   0             \\
			   0
		   \end{pmatrix}}^\transpose                                                  \\
		 & \bm{y} =
		\begin{pmatrix}
			8\bm{v}_4 & 0 & 9\bm{v}_4 & \bm{v}_2 & \bm{v}_4 & \bm{v}_5
		\end{pmatrix}
		=
		{\begin{pmatrix}
			 5x^2 + 6x + 2   \\
			 0               \\
			 4x^2 + 10x + 12 \\
			 7x^2 + 4x + 3   \\
			 12x^2 + 4x + 10 \\
			 7x^2 + 5x + 1
		 \end{pmatrix}}^\transpose
	\end{align*}
	Recall that a possible solution to the R1CS was:
	\[\bm{s} = \begin{pmatrix} 1 & 2 & 3 & 4 & 12 & 10 \end{pmatrix}\]
	Let's check if it is also a valid assignment for the QAP\@:
	\begin{align*}
		p	    = \left(\bm{v}\bm{s}\right)\left(\bm{w}\bm{s}\right) - \bm{y}\bm{s}
		 & = \left(3x^2 + 6x + 6\right)\left(7x^2 + 5x + 3\right) - \left(12x^2 + 7x + 11\right) \\
		 & = 8x^4 + 5x^3 + 4x^2 + 2x + 7
	\end{align*}
	\[
		h = \frac{p}{t} = \frac{8x^4 + 5x^3 + 4x^2 + 2x + 7}{x^3 + 7x^2 + 11x + 7} =
		8x - 51 + \frac{273x^2 + 507x + 364}{x^3 + 7x^2 + 11x + 7} = 8x + 1
	\]
	Since:
	\[ht = \left(8x + 1\right)\left(x^3 + 7x^2 + 11x + 7\right) = 8x^4 + 5x^3 + 4x^2 + 2x + 7 = p\]
	this means that \(p\) and \(h\) are a solution to the QAP, and \(\bm{s}\) is a valid
	assignment.
\end{example}

\noindent One might wonder how a solution \(\left(p, h\right)\) to a QAP is more succint than the
corresponding valid assigment \(\bm{s}\) of the associated R1CS\@: as a matter of fact, given an
\(n/m\) arithmetic circuit, \(\bm{s}\) has size \(n+m+1\), while \(p\) can have degree
(and therefore encoding size) \(2\left(n-1\right)\).
Furthermore, in a typical circuit, \(n \gg m\), so \(\left(p, h\right)\) would approximately be 
twice the size of \(\bm{s}\) when encoded.
Now, \(p = ht \implies \forall x\colon p\left(x\right) = h\left(x\right)t\left(x\right)\); 
if we are working over a big field (say, \(\left|\mathbb{F}\right| \approx 2^{256}\)), it is hard 
to find even a single value of \(x\) for which the equation holds.
This means that we can accept as a solution, with high confindence (although not certainity)
any couple of values \(x, y\) such that \(y \bmod t\left(x\right) = 0\).

Summing up: if \(y \bmod t\left(x\right) = 0\), we are \emph{almost sure} that \(y\) has been 
derived by computing \(p\left(x\right)\), where \(p\) is a solution to our QAP\@.
But if \(p\) is a solution to the QAP, then it derives from a valid assignment \(\bm{s}\) to the
associated R1CS, which in turn derives from a valid computation of the original arithmetic circuit.
