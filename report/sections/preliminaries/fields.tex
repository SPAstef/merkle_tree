\subsection{Fields and groups}
While computational models tipically operate over binary strings, that is, elements of
\({\left\{0, 1\right\}}^*\), where \(*\) indicates Kleene's closure, we often want to interpret
such strings as elements of some algebraic structure.
\begin{definition}
	A \emph{field} is any triple \(\mathbb{F} = \left(F, \oplus, \otimes\right)\), where \(F\) is
	called \emph{underlying set}, \(\oplus\colon F \times F \mapsto F\) is called
	\emph{field addition} and \(\otimes\colon F \times F \mapsto F\) is called
	\emph{field multiplication}, such that both addition and multiplication are commutative and
	associative, multiplication distributes over addition, \(F\) contains an additive identity
	element \(0_{\mathbb{F}}\) and a multiplicative identity element \(1_{\mathbb{F}}\),
	\(\forall x \in F\) there is an additive inverse element \(-x\), and
	\(\forall x \in F\setminus \left\{0_{\mathbb{F}}\right\} \) there is a multiplicative inverse
	element \(x^{-1}\).
\end{definition}

\noindent For example, \(\mathbb{R}\) and \(\mathbb{C}\) are fields, as is Boole's algebra
\(\mathbb{B} = \left(\left\{0, 1\right\}, \textsc{xor}, \textsc{and}\right)\).
We denote elements of a field \(\mathbb{F}\) (abusing notation, as they are actually
elements of the underlying set \(F\)) with lowercase letters \(a, b, c\dots \) and variables over
\(\mathbb{F}\) with lowercase letters \(x, y, z\dots \).

If \(\left|\mathbb{F}\right| \in \mathbb{N}\), then \(\mathbb{F}\) is a \emph{finite field}.
We are particularly interested in finite fields of the kind:
\[\mathbb{F}_{p^k} = \left(\left\{0, \dots, p^k-1\right\}, \oplus_p, \otimes_p\right)\]
where \(p\) is a prime and \(\oplus_p, \otimes_p\) denote addition and
multiplication modulo \(p\) (for example, \(\mathbb{B} = \mathbb{F}_2\)).
Tipically, we consider \(n\)-bit strings either as elements of \(\mathbb{F}_{2^n}\) or of
\(\mathbb{F}_{p^1}\), where \(\log_2\left(p\right) \approx n\).
We will often use \(+\) in place of \(\oplus_p \) when denoting addition and omit \(\otimes_p \)
when denoting multiplication, if \(\mathbb{F}\) is clear from the context.

Any field \(\mathbb{F}\) can be extended to an \(n\)-dimensional vector space
\(\mathbb{F}^n\), for some \(n \in \mathbb{N}\).
We denote vectors in \(\mathbb{F}^n\) with lowercase bold letters (\(\bm{v}, \bm{w}, \dots \)),
and the \(i\)th element of a vector \(\bm{v}\) with \(\bm{v}_i\).
Vector operations follow their natural definitions depending on the underlying field.
We can also introduce matrices over \(\mathbb{F}^{n \times m}\) for some \(n, m \in \mathbb{N}\),
which we denote with bold capital letters (\(\bm{A}, \bm{B}, \dots \)).
The \(i\)th row of a matrix \(\bm{M}\) is denoted with \(\bm{M}_i\), and the \(j\)th element of
the \(i\)th row is denoted with \(\bm{M}_{i,j}\).
Matrix operations also follow their natural definitions over the underlying field.
Given \(\bm{A} \in \mathbb{F}^{n \times m}, \bm{B} \in \mathbb{F}^{n \times m'}\),
we denote with \(\begin{pmatrix}\bm{A} & \bm{B}\end{pmatrix}\) their concatenation along the rows,
and with \(\begin{pmatrix}\bm{A}; \bm{B}\end{pmatrix} =
{\begin{pmatrix}{\bm{A}}^{\transpose} & {\bm{B}}^{\transpose}\end{pmatrix}}^{\transpose}\)
their concatenation along the columns.

A field \(\mathbb{F}\) can also be extended to a monovariate polynomial ring \(\mathbb{F}[x]\),
we will denote polynomials with lowercase letters (\(p, q, \dots \)).
Operations over polynomials are naturally derived from the underlying field.
Vectors and matrices of polynomials are denoted with the usual notation
(\(\bm{p}, \bm{q}, \dots \) and \(\bm{P}, \bm{Q}, \dots \)).

Given some \(\bm{x}, \bm{y} \in \mathbb{F}^n\), we can build the unique polynomial:
\[
	p \mid {p \in \mathbb{F}\left[x\right]} \land {\deg\left(p\right) = n-1} \land
	{\forall i\colon p\left(\bm{x}_i\right) = \bm{y}_i}
\]
by using Lagrange interpolation:
\[
	p = L\left(\bm{x}, \bm{y}\right) =
	\sum_{i}{\bm{y}_{i}\prod_{j \neq i}{\frac{x - \bm{x}_j}{\bm{x}_i - \bm{x}_j}}}
\]
We can extend Lagrange interpolation to any pair of matrices
\(\bm{X}, \bm{Y} \in \mathbb{F}^{n\times m}\) by applying \(L\) to every row:
\[
	L\left(\bm{X}, \bm{Y}\right) =
	\left(L\left(\bm{X}_1, \bm{Y}_1\right) \dots, L\left(\bm{X}_n, \bm{Y}_n\right)\right)
\]

\begin{definition}
	A \emph{group} is a pair \(\mathbb{G} = \left(G, \odot\right)\), where \(G\) is called
	\emph{underlying set}, and \(\odot\colon G \times G \mapsto G\) is called \emph{group composition},
	such that composition is associative, there is a compositive identity element \(1_{\mathbb{G}}\), and
	\(\forall x \in \mathbb{G}\) there is a compositive inverse \(x^{-1}\).
\end{definition}

\noindent For example, \(\mathbb{Z}\) with only addition is a group (where \(1_{\mathbb{Z}} = 0\)),
as is \(\mathbb{Q}\) with only multiplication and without \(0\) (as it's not invertible).
We denote elements and variables over groups in the same way we do for fields.
We can define \emph{group exponentiation} following the natural definition:
\[\forall x \in \mathbb{G}, \forall n \in \mathbb{N}\colon x^n = \bigodot^{n}{x}\]

Any group \(\mathbb{G}\) for which \(\left|\mathbb{G}\right| \in \mathbb{N}\) is a \emph{finite group}.
We can build a finite group from a \emph{generator} set \(S\) and a composition operation
\(\odot \) by closing \(S\) under \(\odot \):
\begin{align*}
	 & {\left\langle{S}\right\rangle}^i =
	\begin{cases}
		S                                            & i = 0
		\\
		{\left\langle{S}\right\rangle}^{i-1} \cup \left\{x \odot y \mid x,y \in
		{\left\langle{S}\right\rangle}^{i-1}\right\} & i > 0
		\\
	\end{cases}
	\\
	 & \mathbb{G} = \left\langle{S}\right\rangle =
	\left(\min_{i}{{\left\langle{S}\right\rangle}^{i}} \mid
	{\left\langle{S}\right\rangle}^{i} = {\left\langle{S}\right\rangle}^{i-1}, \odot\right)
\end{align*}
We are particularly interested in \emph{cyclic groups}, i.e.\ finite groups of the type
\(\mathbb{G}_q\left(g\right) = \langle{g}\rangle = \left({\left\{g^i \bmod q\right\}}_{i \in
	\mathbb{N}}, \otimes_q\right)\), where \(g \in \mathbb{F}_p\) is called \emph{generator element}
and \(\mathbb{F}_p\) is called \emph{underlying field}.
Since every element of \(\mathbb{G}_q\left(g\right)\) is obtained by exponentiating \(g\), we can
define a bijective \emph{discrete logarithm} function:
\[
	\log_g\colon \mathbb{G}_q\left(g\right) \mapsto \mathbb{F}_p =
	\left\{\left(x, y\right) \mid x = g^y\right\}
\]
Cyclic groups are very interesting because, while it's easy to compute exponentiation, no
deterministic algorithm is known that can efficiently compute the discrete logarithm.
\begin{definition}
	Given cyclic groups \(\mathbb{G}_1, \mathbb{G}_2, \mathbb{G}_{\transpose}\), a
	\emph{bilinear map} is any function:
	\[
		B\colon \mathbb{G}_1 \times \mathbb{G}_2 \mapsto \mathbb{G}_{\transpose} \mid
		\forall x \in \mathbb{G}_1, \forall y \in \mathbb{G}_2, \forall a,b \in \mathbb{Z}\colon
		B\left(x^a, y^b\right) = {B\left(x, y\right)}^{ab}
	\]
	If \(\left|\mathbb{G}\right| = \left|\mathbb{G}'\right|\), then \(B\) is \emph{order-preserving},
	and if \(\mathbb{G}_1 = \left\langle{g_1}\right\rangle \land \mathbb{G}_2 =
	\left\langle{g_2}\right\rangle \land
	\mathbb{G}_{\transpose} = \left\langle{B\left(g_1, g_2\right)}\right\rangle \), then \(B\) is
	\emph{non-trivial}.
\end{definition}

\noindent We are interested in order-preserving, non-trivial bilinear maps where
\(\mathbb{G}_1 = \mathbb{G}_2 = \mathbb{G}\).
Bilinar maps have many application in cryptography, as they are used to exploit the hardness of
the discrete logarithm problem.
