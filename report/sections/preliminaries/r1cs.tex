\subsection{Rank-1 Constraint systems}
Like it happens for boolean formulae and the famous \textsc{sat} problem, arithmetic circuits can
also be seen as a form of constraint whose solution is a set of valid assignments for all the
intermediate values in the computation.
\begin{definition}[Rank-1 Contraint System]
	Given a field \(\mathbb{F}\) and some \(m, n \in \mathbb{N}\), a
	\emph{\(n/m\) Rank-1 Constraint System (R1CS)} over \(\mathbb{F}\) is any triple:
	\[\mathcal{C} = (\bm{A}, \bm{B}, \bm{C}) \mid \bm{A}, \bm{B}, \bm{C} \in \mathbb{F}^{n \times m} \]
	A \emph{solution} to some R1CS \(\mathcal{C}\) is any (column) vector:
	\[\bm{s} \mid \bm{s} \in \mathbb{F}^m \land
		(\bm{A}\bm{s})(\bm{B}\bm{s}) = \bm{C}\bm{s} \]
\end{definition}

\noindent Any explicit arithmetic circuit with \(n\) multiplicative gates and \(m\) variables
\(x_1, \dots, x_n\) can be associated with an \(n/(n+m+1)\) R1CS \(\mathcal{C}\) which represents
the constraints in the circuit, roughly in the following way:
\begin{enumerate}
	\item Add a new `constant variable' which always assumes value \(1\).
	\item For every multiplicative gate \(\otimes_i \) in the circuit, add a new \emph{intermediate}
	      variable \(t_i\) (\(t_n\) can be denoted \(y\) as it represents the circuit output).
	\item Define the column vector
	      \(\bm{x} = {\begin{pmatrix}1 & x_1 & \cdots & x_m & t_1 & \cdots & t_n
	      \end{pmatrix}}^\transpose \).
	\item Express every multiplication gate \(\otimes_i \) as an equation in the canonical form:
	      \[(\bm{a_i}\bm{x})(\bm{b_i}\bm{x}) = \bm{c_i}\bm{x}\]
	      where \(\bm{a_i}\) \(\bm{b_i}\) \(\bm{c_i}\) will be the \(i\)th rows of
	      \(\mathcal{C}_{\bm{A}}\), \(\mathcal{C}_{\bm{B}}\) and \(\mathcal{C}_{\bm{C}}\) respectively.
\end{enumerate}

\noindent Let's make an example to better understand the process.
\begin{example}\label{ex:r1cs}
	Consider the explicit arithmetic circuit over \(\mathbb{F}_{13}\) of Example~\ref{ex:circuit}:
	\[\widehat{\phi} = x_2(x_1x_1x_1 + x_2 + x_2 + x_2 + x_2 + 5)\]
	We can see that there are a total of 3 multiplications in the circuit, and since we have two input
	variables, our associated R1CS will be a \(3/6\) R1CS (\(2+1+3 = 6\)).
	Let's explicit all of the intermediate variables:
	\begin{align*}
		 & t_1 = x_1x_1 &  & t_2 = t_1x_1 + 4x_2 + 5 &  & y = t_2x_2
	\end{align*}
	So, our variable vector will be:
	\[\bm{x} = \begin{pmatrix}1 & x_1 & x_2 & t_1 & t_2 & y\end{pmatrix}\]
	Now, let's transform all the equations in canonical form:
	\begin{align*}
		 & (x_1)(x_1) = t_1                                                 \\
		 & {(t_1)(x_1) + 4x_2 + 5 = t_2} \iff {(t_1)(x_1) = 8 + 9x_2 + t_2} \\
		 & (x_2)(t_2) = y
	\end{align*}
	Remember that we are working over \(\mathbb{F}_{13}\), so in the second equation, when we bring
	\(4\) and \(8\) to the right side, we have \(-4 \equiv 9 \pmod{13}\) and \(-5 \equiv 8 \pmod{13}\).
	We can now extract our R1CS \(\mathcal{C} = (\bm{A}, \bm{B}, \bm{C})\):
	\[
		\mathcal{C} =
		\left(
		\begin{pmatrix}
				0 & 1 & 0 & 0 & 0 & 0 \\
				0 & 0 & 0 & 1 & 0 & 0 \\
				0 & 0 & 0 & 0 & 1 & 0
			\end{pmatrix},
		\begin{pmatrix}
				0 & 1 & 0 & 0 & 0 & 0 \\
				0 & 1 & 0 & 0 & 0 & 0 \\
				0 & 0 & 1 & 0 & 0 & 0
			\end{pmatrix},
		\begin{pmatrix}
				0 & 0 & 0 & 1 & 0 & 0 \\
				8 & 0 & 9 & 0 & 1 & 0 \\
				0 & 0 & 0 & 0 & 0 & 1
			\end{pmatrix}
		\right)
	\]
	By construction, a vector \(\bm{s}\) is a solution to \(\mathcal{C}\) \emph{iff} every element
	of \(\bm{s}\) is assigned to the value derived by fixing \(x_1, x_2\) and following the
	computation of the original arithmetic circuit.
	For example, if \(x_1 = 2, x_2 = 3\), we have:
	\begin{align*}
		 & t_1 = x_1x_1 = 2 \times 2 = 4                              &  & \equiv  4  \pmod{13} \\
		 & t_2 = t_1x_1 + 4x_2 + 5 = 4 \times 2 + 4 \times 3 + 5 = 25 &  & \equiv 12 \pmod{13}  \\
		 & y = t_2x_2 = 12 \times 3 = 36                              &  & \equiv 10 \pmod{13}
	\end{align*}
	Therefore our solution vector will be:
	\[\bm{s} = \begin{pmatrix}1 & 2 & 3 & 4 & 12 & 10\end{pmatrix} \]
	It is a bit tedious, but easy, to verify that indeed
	\((\bm{A}\bm{s})(\bm{B}\bm{s}) = \bm{C}\bm{s}\).
\end{example}
