\subsection{Computational models and complexity}
A \emph{computational model} (or model of computation) is any kind of `device', either physical or
mathematical, which is able to compute algorithms to solve problems.
A particularly interesting class of problems are \emph{decision problems}, the ones that can be
answered with `yes' (or `accept', or \(\top \)) or `no' (or `reject', or \(\bot \)).
Every computational model is able to \emph{decide} only a subclass of all decision problems, and
even then, not all can be solved \emph{efficiently}, that is, by using an amount of resources
(tipically, time and space) which is upper-bounded by some polynomial function of the input length.
Problems for which a polynomial bound doesen't exist or isn't known are said to be \emph{hard} for
the computational model.
For example, finding solutions to boolean equations (the \textsc{sat} problem) is believed to be
hard for deterministic Turing machines, but it is easy for non-deterministic ones.
Unfortunately, non-deterministic Turing machines (along with any other non-deterministic model of
computation) are more of a mathematical tool than anything, and there seems to be no practical way
to efficiently solve the problems which would take Non-deterministic Polynomial time (NP-complete 
problems) as stated by the strong Church-Turing thesis.
While it is widely belived that efficiently solving NP-complete problems is impossible, there are 
some problems which lie in a `gray zone' between \textsc{NP-complete} and \textsc{P} (i.e.\ 
problems which can be solved in deterministic polynomial time): they are believed to be hard for 
deterministic models (hence, they are not in \textsc{P}), but there is no proof that they are 
\textsc{NP-complete}.
The most famous of such problems is factorization: with the advent of quantum-computing, 
which challenges the strong Church-Turing thesis, Shor devised an efficient quantum algorithm 
for factorizing numbers.
While still far from usable in practical cases, its existence proves that one must be extremely
careful when talking about the hardness of problems, especially when applied to cryptography, 
and must always make clear assumptions on the underlying model of computation.
