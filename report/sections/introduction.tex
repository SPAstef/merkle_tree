\section{Introduction}
One of the biggest revolutions of the last decade has been the widespread adoption of
blockchain-based technologies~\cite{NarayananBFMG2016}. 
For example, cryptocurrencies like Bitcoin or Ethereum are widely known even among the 
non-crypto-enthusiasts and a huge market is growing around them.
There are other highly interesting applications for the blockchain outside the cryptocurrency world:
in fact, anything which requires some degree of `verifiability' in a non-trusted environment can
benefit from the usage of a blockchain technology.
Typically, a block of the chain does not correspond to a single transaction, as the blockchain
would grow too quickly for practical storage: many transactions are inserted into a tree 
data-structure, called Merkle Tree (MT)~\cite{Merkle1988}, which is computed bottom-up and whose 
root is then inserted into the blockchain.
In a traditional setup (e.g.\ cryptocurrencies), all the data which is required to build a block of 
the blockchain is of public knowledge: when a transaction happens, the details of that transaction 
are shared with the network for it to be validated.
There are however many scenarios where one would like for the data to remain secret, as there could
be risks involving privacy violation (e.g.\ transactions) or intellectual property theft 
(e.g.\ shared computation).

Zero-Knowledge Succinct Non-interactive ARgument of Knowledge\\ (ZK-SNARK) systems are cryptographic
frameworks which allow for a party to convince other parties about the knowledge of some secret 
without revealing the secret.
As a simple example, one would be able convince other parties that a transaction has been performed 
honestly, but without revealing the details of the transaction.
ZK-SNARK associate the claim made by the challenger with an instance of a problem which 
is almost impossible to solve if the secrets involved in the claim are not actually true.
Since ZK-SNARK over prime fields and groups, and traditional hashing algorithms, like 
SHA-256~\cite{Dang2015}, are not efficient when adapted for these frameworks.
For this reason, new hashing algorithms like MiMCHash~\cite{AlbrechtGRRT2016} have been designed 
with the ZK scenario in mind, and the Augmented Binary tRee (ABR)~\cite{AndreevaBR2021} data 
structure tries to improve on the compactness of Merkle Trees by processing more transactions with 
the same amount of calls to the underlying hash function.

\subsection*{Organization of this report}
In Section~\ref{sec:preliminaries}, we introduce some basic notions of computational and complexity
theory, group and field theory, hash functions and tree hash modes; we also introduce arithmetic
circuits, Rank-1 Constraint Systems and Quadratic Arithmetic Programs, which are an essential tool
for ZK-SNARK systems.
In Section~\ref{sec:zksnark}, we introduce ZK-SNARK, and study the Pinocchio protocol, as well
as the ZK-SNARK-friendly MiMC permutation.
In Section~\ref{sec:experiments}, we compare our implementations of MiMC and ABRs with SHA and
Merkle Trees using the \texttt{libsnark} library.
Finally, in Section~\ref{sec:conclusions} we draw our conclusions and discuss possible future
directions.
