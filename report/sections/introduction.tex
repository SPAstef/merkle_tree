\section{Introduction}
One of the biggest revolutions of the last decade has been the widespread adoption of
blockchain-based technologies. In particular, cryptocurrencies like Bitcoin or Ethereum are
widely known even among the non-crypto-enthusiasts and a huge market is growing around them.
There are highly interesting applications of the blockchain even outside the financial world:
in fact, anything which requires some degree of `verifiability' in a non-trusted environment can
benefit from using a blockchain.
Tipically, a block of the chain does not correspond to a single transaction, as the blockchain
would become too big to be stored: many transactions are inserted into a tree data-structure,
called Merkle Tree (MT), which is computed bottom-up and whose root is then inserted into the blockchain.
In a traditional setup (e.g.\ the aformentioned cryptocurrencies), all the data which is required
to build a block of the blockchain is of public knowledge: when an Ethereum transaction happens,
the details of that transaction are shared with the network for it to be validated.
Of course, there are many scenarios were one would like the data to remain secret, as there could
be risks for privacy (e.g.\ transactions) or for intellectual property (e.g.\ algorithms).

Zero-Knowledge Succint Non-interactive ARgument of Knowledge\\ (ZK-SNARK) systems are cryptographic
frameworks which allow for a party to convince other parties that he `knows something' without
revealing anything else.
For example, one could convice other users that the hash of a transaction is valid without
revealing the details of that transaction.
ZK systems work over prime fields, and hashing algorithms like SHA-256, which are quite efficient
in a `vanilla' scenario, can become extremely slow when translated.
For this reason, new hashing algorithms like MiMCHash have been designed with the ZK scenario in
mind, and Augmented Binary tRee (ABR) tries to improve Merkle Trees by processing more transactions
without requiring more hash function calls.

In Section~\ref{sec:preliminaries}, we introduce some basic notions of computational and complexity
theory, group and field theory, hash functions and tree hash modes; we also introduce arithmetic
circuits, Rank-1 Constraint Systems and Quadratic Arithmetic Programs, which are an essential tool
for ZK-SNARK systems.
In Section~\ref{sec:zksnark}, we introduce ZK-SNARK, and study the Pinocchio protocol, as well
as the ZK-SNARK-friendly MiMC permutation.
In Section~\ref{sec:experiments}, we compare our implementations of MiMC and ABRs with SHA and
Merkle Trees using the \texttt{libsnark} library.
Finally, in Section~\ref{sec:conclusions} we draw our conclusions and discuss possible future
directions.
