\subsection{Zero Knowledge Proofs}
Before diving into provable computation, we must introduce the more general concept of Zero
Knowledge Proof system.
\begin{definition}[Zero-Knowledge Proof~\cite{Goldreich2002}]
	Given a prover \(\mathcal{P}\) and a verifier \(\mathcal{V}\), a secret \(x\), known only to
	\(\mathcal{P}\), and some statement \(\sigma \) of whose truth \(\mathcal{P}\) wants to convince
	\(\mathcal{V}\) by means of some proof \(\pi \), we call a Zero-Knowledge Proof (ZKP) system any
	protocol which satisfies the following properties:
	\begin{itemize}
		\item \textbf{Soundness}: \(\neg{\sigma} \implies \mathcal{V}\left(\pi\right) = \bot \).
		\item \textbf{Completeness}: \(\sigma \implies \mathcal{V}\left(\pi\right) = \top \).
		\item \textbf{Zero-Knowledge}: It is \emph{hard} for \(\mathcal{V}\) to derive \(x\) given
		      \(\sigma \) and \(\pi \).
	\end{itemize}
\end{definition}

\noindent While formal proofs have been known for millennia, only in the last century, with the
advent of modern cryptography, researchers started considering the possibility of having proofs
of statements which, while able to convince someone of their truth, didn't leak information
about how they were obtained.
Zero-Knowledge systems proves particularly useful in \emph{ARgument of Knowledge} scenarios
(ZK-ARK): the prover \(*\) wants to convince the verifier \(\mathcal{V}\) that he
knows a solution to some problem, assuming there is one, without revealing it.
It must be noted though that known ZK-ARK systems do not guarantee the formal soundness of
the proof: there is a small probability that, given some false statement \(\sigma \) and an
(invalid) proof \(\pi \), then \(\mathcal{V}\left(\pi\right) = \top \), so it is important to keep
this probability small (say, \(2^{-128}\)).
\begin{definition}[ZK-SNARK~\cite{SassonCTV2013}]
	Given a prover \(\mathcal{P}\), a verifier \(\mathcal{V}\), a statement \(\sigma \), and a proof
	\(\pi \), a Zero-Knowledge Non-interactive ARgument of Knowledge (ZK-SNARK) system is any
	ZK-ARK system which is:
	\begin{itemize}
		\item \textbf{Succinct}: \(SPACE\left(\pi\right) =
		      \SmallO\left(\log\left(\sigma\right)\right)\).
		\item \textbf{Non-interactive}: The only communication required by the system is the exchange
		      of \(\sigma \) and \(\pi \).
	\end{itemize}
\end{definition}

\noindent Succinctness is an important property in many scenarios, like blockchains, since we
cannot afford to use too much resources to transmit and store the proofs, and non-interactivity of
the process allows for efficient verification when multiple parties are involved.

One of the most important applications of ZK-SNARK systems is in \emph{provable computation},
where the prover wants to convince the verifier that he correctly performed some computation
(e.g.\ a cryptocurrency transaction).
