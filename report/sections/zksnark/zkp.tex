\subsection{Zero Knowledge Proofs}
Before diving into provable computation, we must introduce tghe more general concept of Zero
Knowledge Proof system.
\begin{definition}[Zero-Knowledge Proof]
	Given a prover \(P\) and a verifier \(V\), a secret \(x\), known only to \(P\), and some
	statement \(\sigma \) of whose truth \(P\) wants to convince \(V\) by means of
	some proof \(\pi \), we call a Zero-Knowledge Proof (ZKP) system any protocol which satisfies
	the following properties:
	\begin{itemize}
		\item \textbf{Soundness}: \(\neg{\sigma} \implies V(\pi) = \bot \).
		\item \textbf{Completeness}: \(\sigma \implies V(\pi) = \top \).
		\item \textbf{Zero-Knowledge}: It is \emph{hard} for \(V\) to derive \(x\) given \(\sigma \)
		      and \(\pi \).
	\end{itemize}
\end{definition}

\noindent While formal proofs have been known for millenia, only in the last century, with the
advent of modern cryptography, researchers started considering the possibility of having proofs
of statements which, while able to convice someone of their truth, didn't leak information
about how they were obtained.
Zero-Knowledge systems proves particularly useful in \emph{ARgument of Knowledge} scenarios
(ZK-ARK): the prover \(P\) wants to convince the verifier \(V\) that he
knows a solution to some computational problem, assuming there is one, without revealing it.
This becomes particularly handy when dealing with hard problems, like committing to some message
\(m\) by computing some secure hash function \(h = H(m)\).
Since \(H\) is a OWF, it would be too hard for \(V\) to `make-up' some fake \(m'\) which hashes 
to \(h\), .
It must be noted though that known ZK-ARK systems do not guarantee the formal soundness of
the proof: there is a small probability that, given some false statement \(\sigma \) and an
(invalid) proof \(\pi \), then \(V(\pi) = \top \), so it is important to keep this probability
small (say, \(2^{-128}\)).
There are other nice additional properties that zero-knowledge systems might satisfy, making
them even more interesting.
\begin{definition}[ZK-SNARK]
	Given a prover \(P\), a verifier \(V\), a statement \(\sigma \), and a ZK-ARK system to produce
	a proof \(\pi \), if the system is:
	\begin{itemize}
		\item \textbf{Succint}: \(SPACE(\pi) = \SmallO(\log(\sigma))\).
		\item \textbf{Non-interactive}: The only communication required by the system is the exchange
		      of \(\sigma \) and \(\pi \).
	\end{itemize}
	then it is a Zero-Knowledge Non-interactive ARgument of Knowledge (ZK-SNARK) system.
\end{definition}

\noindent Succintness is an important property in a blockchain scenario, since we cannot
afford to use too much space to store the proofs, and non-interactivity of the process allows for
efficient verification when multiple parties are involved.

One of the most important applications of ZK-SNARK systems is in \emph{provable computation},
where the prover wants to convince the verifier that he correctly performed some computation
(e.g.\ a cryptocurrency transaction).
A very famous ZK-SNARK system for verifiable computation is the \emph{Pinocchio} protocol,
which was the first one efficient enough to be practical.
