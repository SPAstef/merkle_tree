\subsection{CHFs for ZK-SNARK systems}
We mentioned that Pinocchio, like other ZK-SNARK systems, works on big prime fields.
The most famous CHFs, like MD5~\cite{Rivest1990} or the SHA families, make extensive use of
bit-wise operations, basically working on some field of the type \(\mathbb{F}_{2^n}\).
When translating these hash functions as arithmetic circuits over a prime field \(\mathbb{F}_p\),
we incur in a huge space and time overhead.
\begin{example}
	Consider some \(S, T \in {\left\{0, 1\right\}}^n\) for some \(n \in \mathbb{N}\), and assume we
	want to compute \(S \mathbin{\textsc{xor}} T\).
	If we work over \(\mathbb{F}_{2^n}\) the resulting arithmetic circuit would simply be
	\(x \oplus y\), with \(x = \sum_{i}{S_{i}2^{i-1}}\) and \(y = \sum_{i}{T_{i}2^{i-1}}\).

	On the other hand, to express the same operation over \(\mathbb{F}_p\) we consider \(S\) and
	\(T\) as vectors
	\(\bm{x}, \bm{y} \in \mathbb{F}^n \mid \forall i\colon \bm{x}_i = S_i \wedge \bm{y}_i = T_i\).
	For a single-bit \textsc{xor} operation, we use the circuit
	\(x \oplus y - 2\left(x \otimes y\right) \); to represent an n-bit \textsc{xor}, we concatenate
	\(n\) such circuits.
	What required a single addition in \(\mathbb{F}_{2^n}\) requires \(3n\) additions and \(n\)
	multiplications in \(\mathbb{F}_p\)!

	Furthermore, since \(\bm{x}\) and \(\bm{y}\) are defined over \(\mathbb{F}_p^n\), we must also
	guarantee that their elements are either \(0\) or \(1\), by explicitly adding the constraint
	\(\left(x\right) \otimes \left(x - 1\right) = 0\) for every \(x \in \bm{x}, \bm{y}\).
\end{example}

\noindent Having efficient CHFs for ZK systems is extremely important, as they are the fundamental
building block of commitment protocols which are in turn the pivot of the secure transaction
protocols used by cryptocurrencies like ZCash\footnote{\url{https://z.cash/}}.
For this reason, we want CHFs which use native operations over \(\mathbb{F}_p\):
while they are typically much slower when implemented, say, in x86 assembly, they become
immensely faster in ZK-SNARK frameworks due to their extremely lower multiplicative complexity.
\begin{definition}[MiMC primitives~\cite{AlbrechtGRRT2016}]
	Given a finite field \(\mathbb{F}_p\),
	\(r = \left\lceil{\frac{\log_2\left(p\right)}{\log_2\left(3\right)}}\right\rceil \), some
	\(\bm{c} \in \mathbb{F}_p^r\) and the functions:
	\[\forall i < r\colon f_i\left(x,k\right)\colon \mathbb{F}_p \times \mathbb{F}_p \mapsto
		\mathbb{F}_p = x^3 \oplus k \oplus \bm{c}_i\]
	the \emph{MiMC keyed permutation} is defined as:
	\[E\left(x, k\right)\colon \mathbb{F}_p \times \mathbb{F}_p \mapsto \mathbb{F}_p =
		\left(f_{r} \circ \dots \circ f_1\right)\left(x, k\right) \oplus k\]
	By fixing \(k = 0\), we obtain the \emph{MiMC permutation} \(P\left(x\right)\).
	By applying the Davies-Meyer and the Merkle-Damg\r{a}rd constructions to the MiMC permutation, we 
	obtain the \emph{MiMC hash function} \(H\left(x\right)\).
\end{definition}
