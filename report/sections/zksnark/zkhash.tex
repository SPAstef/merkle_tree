\subsection{CHFs for ZK-SNARK systems}
We mentioned that Pinocchio, like other ZK-SNARK systems, works on big prime fields.
The most famous CHFs, like MD5 or the SHA families, make extensive use of bitwise operations,
basically working on some field of the type \(\mathbb{F}_{2^n}\).
When translating these hash functions as arithmetic cirtuits over a prime field \(\mathbb{F}_p\),
we incur in a huge space and time overhead.
\begin{example}
	Consider some \(S, T \in {\{0, 1\}}^n\) for some \(n \in \mathbb{N}\), and assume we want to
	compute \(S \mathbin{\textsc{xor}} T\).
	If we work over \(\mathbb{F}_{2^n}\) the resulting arithmetic circuit would simply be
	\(x \oplus y\), with \(x = \sum_{i}{S_{i}2^{i-1}}\) and \(y = \sum_{i}{T_{i}2^{i-1}}\).

	On the other hand, to express the same operation over \(\mathbb{F}_p\) we consider \(S\) and
	\(T\) as vectors
	\(\bm{x}, \bm{y} \in \mathbb{F}^n \mid \forall i\colon \bm{x}_i = S_i \wedge \bm{y}_i = T_i\).
	For a single-bit \textsc{xor} operation, we use the circuit \(x \oplus y - 2(x \otimes y) \);
	to represent an n-bit \textsc{xor}, we concatenate \(n\) such circuits.
	What required a single addition in \(\mathbb{F}_{2^n}\) requires \(3n\) additions and \(n\)
	multiplications in \(\mathbb{F}_p\)!

	Furthermore, since \(\bm{x}\) and \(\bm{y}\) are defined over \(\mathbb{F}_p^n\), we must also
	guarantee that their elements are either \(0\) or \(1\), by explicitly adding the contraint
	\((x)\otimes(x - 1) = 0\) for every \(x \in \bm{x}, \bm{y}\).
\end{example}

For this reason, we want CHFs which use native operations over \(\mathbb{F}_p\):
while they are tipically much slower if implemented, say, in x86 assembly, they become
immensely faster when implemented in a ZK-SNARK framework.

\begin{definition}[MiMC primitives]
	Given a finite field \(\mathbb{F}_p\), 
  \(r = \left\lceil{\frac{\log_2(p)}{\log_2(3)}}\right\rceil \), some \(\bm{c} \in \mathbb{F}_p^r\) 
  and the functions:
	\[\forall i < r\colon f_i(x,k)\colon \mathbb{F}_p \times \mathbb{F}_p \mapsto \mathbb{F}_p =
		x^3 \oplus k \oplus \bm{c}_i\]
	the \emph{MiMC keyed permutation} is defined as:
	\[E(x, k)\colon \mathbb{F}_p \times \mathbb{F}_p \mapsto \mathbb{F}_p =
		(f_{r} \circ \dots \circ f_1)(x, k) \oplus k\]
	By fixing \(k = 0\), we obtain the \emph{MiMC permutation} \(P(x)\).
	By applying the Davies-Meyer construction to the MiMC permutation, we obtain the
	\emph{MiMC compression function} \(C(x)\).
	Finally, by applying the Merkle-Damg\r{a}rd construction to the MiMC compression function, we
	obtain the \emph{MiMC hash function} \(H(x)\).
\end{definition}

