\subsection{The Pinocchio Protocol}
A very famous ZK-SNARK system for verifiable computation is the \emph{Pinocchio} protocol,
which was the first one efficient enough to be practical.
Pinocchio uses a lot of mathematical machinery, while we tried to introduce most of the required
background in Section~\ref{sec:preliminaries}, it will still be not trivial to understand fully
\emph{how}, and, more importantly, \emph{why}, it actually works.
We will not go into all of the deep details of the protocol, especially for what concernes the
last stages, but we will still try to explain it in all its parts.

First of all, Pinocchio does not allow the encoding of arbitrary languages, i.e.\ it is not Turing
complete, but we are restricted to arithmetic circuits over some big prime field \(\mathbb{F}_p\)
(tipically, \(p \approx 2^{256}\)).
The main limitation arising from this restriction is that we can only express bounded computation
(i.e.\ no loops whose length depends on some variable condition).
This issue can be mitigated by writing a \emph{circuit synthesizer} in a Turing complete language
which is able to build parametrized arithmetic circuits `on the fly'.
Given an arithmetic circuit \(\phi \) over a prime field \(\mathbb{F}_p\), Pinocchio works as
follows:
\begin{enumerate}
	\item Fix some generator \(g \in \mathbb{F}_p\) such that \(\mathbb{G} = \langle{g}\rangle \) and an
	      order-preserving non-trivial bilinear map
	      \(B\colon \mathbb{G} \times \mathbb{G} \mapsto \mathbb{G}_{\transpose}\).
	\item Build the R1CS \(\mathcal{C}\) associated with \(\phi \).
	\item Build the QAP \(\mathcal{Q}\) associated with \(\mathcal{C}\).
	\item A trusted third party \(\mathcal{T}\) generates a set of random elements
	      \(R \in \mathbb{F}_p\) which, together with \(g\), are used to build a \emph{prover key}
	      \(K_{\mathcal{P}} \in \mathbb{G}\) of size \(\Theta(|\phi|)\) and a \emph{verifier key}
	      \(K_{\mathcal{V}} \in \mathbb{G}\) of size \(\Theta(|\phi_{IO}|)\) (\(\phi_{IO}\)
	      denotes the set of inputs and outputs of \(\phi \)).
	      The random data is deleted immediately after use (\emph{toxic waste}).
	\item The prover know executes \(\phi \), computes all the intermediate values, and uses them to
	      solve \(\mathcal{C}\) and \(\mathcal{Q}\), finding a solution \((p, h)\) for \(\mathcal{Q}\).
	\item The prover chooses some value \(x \in \mathbb{F}_p\) with which he computes \(p(x)\) and
	      \(h(x)\), and uses \(K_{\mathcal{P}}\) to encrypt them, producing a proof of the kind
	      \(k^{p(x)} = k^{t(x)h(x)}\), which has size \(\Theta(1)\), and is sent to the
	      verifier.
	\item The verifier uses the key \(K_{\mathcal{V}}\), exploiting the bilinear map \(B\), to
	      verify that \(k^{p(x)} = k^{t(x)h(x)}\), which implies that \(p(x) = t(x)h(x)\), which
	      implies with high probability that \(p = th\) which, as we know, implies that
	      \(\phi \) was correctly executed.
\end{enumerate}

\noindent Note how all the work that has to be done from steps 1 to 4 depends only on the
circuit, not on some particular execution, so everything that has been computed in those steps 
can be saved and reused later.
