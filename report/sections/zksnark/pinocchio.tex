
\subsection{The Pinocchio Protocol}
Pinocchio is composed of many different components, and requires quite a bit of mathematical
background to be fully understood. We will not go into all of the mathematical and cryptographic
details of the protocol, especially the ones involving \emph{elliptic curves},
but we will still try to give a good idea of how the protocol works, and ultimately what
determines its computational complexity.

Pinocchio does not allow the encoding of arbitrary languages, i.e.\ it is not Turing complete, but
we are restricted to arithmetic circuits over an arbitrary prime field \(\mathbb{F}_p\).
The main limitation arising from this restriction is that we cannot express unbounded computation
(e.g.\ loops whose exit condition depends on some non-constant value) in this framework.
This issue can be mitigated by writing a circuit compiler in a Turing complete language which
is able to synthesize parametrized arithmetic circuits on the fly.

\noindent The reason we translate arithmetic circuits into R1CS is that they explicit all of the
computation in terms of linear combinations, which allows us to use Lagrange interpolation to
build polynomials over them.

\noindent
With all this in mind, the flow of the Pinocchio protocol is as follows:
\begin{enumerate}
	\item Encode an algorithm as an arithmetic circuit over some prime field \(\mathbb{F}_p\).
	\item Compute the associated R1CS\@.
	\item Compute the associated QAP\@.
	\item Generate random values and instantiate an homomorphic encryption mapping
	      (using elliptic curves) which depends on those values.
	\item Generate the proof by encrypting the QAP with the homomorphic mapping.
	\item Map again the proof to a new homomorphic space, and finally verify it.
\end{enumerate}

\noindent Due to the homomorphism of the mappings and the properties of QAPs and R1CSs, if the
verification is successful, it means that the original algorithm was in fact correctly executed,
with high probability.
If the verification fails, then certainly the original algorithm was not executed correctly.
The verifier learns cannot learn any additional information from this process without performing
an infeasible amount of work, therefore this is indeed a ZK-SNARK protocol.
