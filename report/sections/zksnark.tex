\section{ZK-SNARK}\label{sec:zksnark}
We saw in Section~\ref{sec:preliminaries} how a prover can convince a verifier about the knowledge of
some message \(m\), with a high confidence and a small communication effort, by using a CHF \(H\).
However, the underlying assumption was that \(m\) is known by the verifier: when the prover sends
\(h\), the verifier can check whether \(H\left(m\right) = h\) and therefore accept or reject.
In this Section, we will see how a prover can convince a verifier without the need to disclose
possibly secret information.
In particular, we will focus on provable computation, that is, when the prover wants to convice
the verifier that he correctly computed some function.

\subsection{Zero Knowledge Proofs}
Before diving into provable computation, we must introduce tghe more general concept of Zero
Knowledge Proof system.
\begin{definition}[Zero-Knowledge Proof]
	Given a prover \(P\) and a verifier \(V\), a secret \(x\), known only to \(P\), and some
	statement \(\sigma \) of whose truth \(P\) wants to convince \(V\) by means of
	some proof \(\pi \), we call a Zero-Knowledge Proof (ZKP) system any protocol which satisfies
	the following properties:
	\begin{itemize}
		\item \textbf{Soundness}: \(\neg{\sigma} \implies V(\pi) = \bot \).
		\item \textbf{Completeness}: \(\sigma \implies V(\pi) = \top \).
		\item \textbf{Zero-Knowledge}: It is \emph{hard} for \(V\) to derive \(x\) given \(\sigma \)
		      and \(\pi \).
	\end{itemize}
\end{definition}

\noindent While formal proofs have been known for millenia, only in the last century, with the
advent of modern cryptography, researchers started considering the possibility of having proofs
of statements which, while able to convice someone of their truth, didn't leak information
about how they were obtained.
Zero-Knowledge systems proves particularly useful in \emph{ARgument of Knowledge} scenarios
(ZK-ARK): the prover \(P\) wants to convince the verifier \(V\) that he
knows a solution to some problem, assuming there is one, without revealing it.
It must be noted though that known ZK-ARK systems do not guarantee the formal soundness of
the proof: there is a small probability that, given some false statement \(\sigma \) and an
(invalid) proof \(\pi \), then \(V(\pi) = \top \), so it is important to keep this probability
small (say, \(2^{-128}\)).
\begin{definition}[ZK-SNARK]
	Given a prover \(P\), a verifier \(V\), a statement \(\sigma \), and a proof \(\pi \), a 
  Zero-Knowledge Non-interactive ARgument of Knowledge (ZK-SNARK) system is any ZK-ARK system 
  which is:
	\begin{itemize}
		\item \textbf{Succint}: \(SPACE(\pi) = \SmallO(\log(\sigma))\).
		\item \textbf{Non-interactive}: The only communication required by the system is the exchange
		      of \(\sigma \) and \(\pi \).
	\end{itemize}
\end{definition}

\noindent Succintness is an important property in many scenarios, like blockchains, since we 
cannot afford to use too much resources to transmit and store the proofs, and non-interactivity of 
the process allows for efficient verification when multiple parties are involved.

One of the most important applications of ZK-SNARK systems is in \emph{provable computation},
where the prover wants to convince the verifier that he correctly performed some computation
(e.g.\ a cryptocurrency transaction).
\subsection{The Pinocchio Protocol}
A very famous ZK-SNARK system for verifiable computation is the \emph{Pinocchio} protocol,
which was the first one efficient enough to be practical.
Pinocchio uses a lot of mathematical machinery, and it's not trivial to fully understand
\emph{how}, and even more importantly, \emph{why}, it actually works.
We will not go into all of the details of the protocol, especially in the last stages which involve
\emph{elliptic curve} mathematics, but we will still try to give a good idea of the first stages
and an intuition of the last ones, focusing on what determines the computational complexity of
this protocol.

Pinocchio does not allow the encoding of arbitrary languages, i.e.\ it is not Turing complete, but
we are restricted to arithmetic circuits over some prime field \(\mathbb{F}_p\).
The main limitation arising from this restriction is that we cannot express unbounded computation
(like infinite loops) or even variably-bounded computation (like loops whose exit condition
depends on some non-constant value).
This issue can be mitigated by writing a \emph{circuit synthesizer} in a Turing complete language
which is able to build parametrized arithmetic circuits `on the fly'.
After we have our arithmetic circuit \(\phi \) over a prime field \(\mathbb{F}_p\), the Pinocchio
works as follows:
\begin{enumerate}
	\item The circuit \(\phi \) is made public.
	\item Build the R1CS \(\mathcal{C}\) associated with \(\phi \).
	\item Build the QAP \(\mathcal{Q}\) associated with \(\mathcal{C}\).
	\item A trusted third party generates some random data, which is used to create a prover key
	      (\(k_P\)) and a verifier key \(k_V\). The random data must be kept secret
	      (or even better, deleted after use).
	\item The prover executes \(\phi \), computes all the intermediate values, and uses them to
	      solve \(\mathcal{C}\) and \(\mathcal{Q}\); in the end, he finds a solution \((p, h)\)
	      for \(\mathcal{Q}\).
	\item The prover chooses some value \(x\) to compute \(p(x)\) and \(h(x)\), and uses \(k_P\)
	      to generate an encrypted proof, of size \(\BigO(1)\), which is sent to the verifier.
	      The encryption scheme exploits group theory so that \(p(x) = h(x)t(x)\) if and only if
	      a different (but still easy) operation involving the encrypted values holds in the
	      encrypted space.
	      This involves using a public cyclic group 
        \(\mathbb{G}_q = ({\{g^i \bmod q\}}_{i \in \mathbb{F}_p}, \otimes)\) which is generated by 
        some public element \(g \in \mathbb{F}_p\) and a 
        prime number \(q \in \mathbb{N}\), together with a bilinear mapping
	      \(B\colon \mathbb{G}_q \times \mathbb{G}_q \mapsto \mathbb{F}_q\) (bilinear means that
	      \(B(x^a, y^b) = {B(x, y)}^{ab}\)).
	      The group \(\mathbb{G}_q\) was also used to generate \(k_P\) and \(k_V\).
	\item
\end{enumerate}

\noindent Due to the homomorphism of the mappings and the properties of QAPs and R1CSs, if the
verification is successful, it means that the original algorithm was in fact correctly executed,
with high probability.
If the verification fails, then certainly the original algorithm was not executed correctly.
The verifier learns cannot learn any additional information from this process without performing
an infeasible amount of work, therefore this is indeed a ZK-SNARK protocol.

\subsection{CHFs for ZK-SNARK systems}
We mentioned that Pinocchio, like other ZK-SNARK systems, works on big prime fields.
The most famous CHFs, like MD5~\cite{Rivest1990} or the SHA families, make extensive use of
bit-wise operations, basically working on some field of the type \(\mathbb{F}_{2^n}\).
When translating these hash functions as arithmetic circuits over a prime field \(\mathbb{F}_p\),
we incur in a huge space and time overhead.
\begin{example}
	Consider some \(S, T \in {\left\{0, 1\right\}}^n\) for some \(n \in \mathbb{N}\), and assume we
	want to compute \(S \mathbin{\textsc{xor}} T\).
	If we work over \(\mathbb{F}_{2^n}\) the resulting arithmetic circuit would simply be
	\(x \oplus y\), with \(x = \sum_{i}{S_{i}2^{i-1}}\) and \(y = \sum_{i}{T_{i}2^{i-1}}\).

	On the other hand, to express the same operation over \(\mathbb{F}_p\) we consider \(S\) and
	\(T\) as vectors
	\(\bm{x}, \bm{y} \in \mathbb{F}^n \mid \forall i\colon \bm{x}_i = S_i \wedge \bm{y}_i = T_i\).
	For a single-bit \textsc{xor} operation, we use the circuit
	\(x \oplus y - 2\left(x \otimes y\right) \); to represent an n-bit \textsc{xor}, we concatenate
	\(n\) such circuits.
	What required a single addition in \(\mathbb{F}_{2^n}\) requires \(3n\) additions and \(n\)
	multiplications in \(\mathbb{F}_p\)!

	Furthermore, since \(\bm{x}\) and \(\bm{y}\) are defined over \(\mathbb{F}_p^n\), we must also
	guarantee that their elements are either \(0\) or \(1\), by explicitly adding the constraint
	\(\left(x\right) \otimes \left(x - 1\right) = 0\) for every \(x \in \bm{x}, \bm{y}\).
\end{example}

\noindent Having efficient CHFs for ZK systems is extremely important, as they are the fundamental
building block of commitment protocols which are in turn the pivot of the secure transaction
protocols used by cryptocurrencies like ZCash\footnote{\url{https://z.cash/}}.
For this reason, we want CHFs which use native operations over \(\mathbb{F}_p\):
while they are typically much slower when implemented, say, in x86 assembly, they become
immensely faster in ZK-SNARK frameworks due to their extremely lower multiplicative complexity.
\begin{definition}[MiMC primitives~\cite{AlbrechtGRRT2016}]
	Given a finite field \(\mathbb{F}_p\),
	\(r = \left\lceil{\frac{\log_2\left(p\right)}{\log_2\left(3\right)}}\right\rceil \), some
	\(\bm{c} \in \mathbb{F}_p^r\) and the functions:
	\[\forall i < r\colon f_i\left(x,k\right)\colon \mathbb{F}_p \times \mathbb{F}_p \mapsto
		\mathbb{F}_p = x^3 \oplus k \oplus \bm{c}_i\]
	the \emph{MiMC keyed permutation} is defined as:
	\[E\left(x, k\right)\colon \mathbb{F}_p \times \mathbb{F}_p \mapsto \mathbb{F}_p =
		\left(f_{r} \circ \dots \circ f_1\right)\left(x, k\right) \oplus k\]
	By fixing \(k = 0\), we obtain the \emph{MiMC permutation} \(P\left(x\right)\).
	By applying the Davies-Meyer and the Merkle-Damg\r{a}rd constructions to the MiMC permutation, we 
	obtain the \emph{MiMC hash function} \(H\left(x\right)\).
\end{definition}

