\section{ZK-SNARK}\label{sec:zksnark}
We saw in Section~\ref{sec:preliminaries} how a prover can convince a verifier about the knowledge of
some message \(m\), with a high confidence and a small communication effort, by using a CHF \(H\).
However, the underlying assumption was that \(m\) is known by the verifier: when the prover sends
\(h\), the verifier can check whether \(H\left(m\right) = h\) and therefore accept or reject.
In this Section, we will see how a prover can convince a verifier without the need to disclose
possibly secret information.
In particular, we will focus on provable computation, that is, when the prover wants to convince
the verifier that he correctly computed some function.

\subsection{Zero Knowledge Proofs}
Before diving into provable computation, we must introduce tghe more general concept of Zero
Knowledge Proof system.
\begin{definition}[Zero-Knowledge Proof]
	Given a prover \(P\) and a verifier \(V\), a secret \(x\), known only to \(P\), and some
	statement \(\sigma \) of whose truth \(P\) wants to convince \(V\) by means of
	some proof \(\pi \), we call a Zero-Knowledge Proof (ZKP) system any protocol which satisfies
	the following properties:
	\begin{itemize}
		\item \textbf{Soundness}: \(\neg{\sigma} \implies V(\pi) = \bot \).
		\item \textbf{Completeness}: \(\sigma \implies V(\pi) = \top \).
		\item \textbf{Zero-Knowledge}: It is \emph{hard} for \(V\) to derive \(x\) given \(\sigma \)
		      and \(\pi \).
	\end{itemize}
\end{definition}

\noindent While formal proofs have been known for millenia, only in the last century, with the
advent of modern cryptography, researchers started considering the possibility of having proofs
of statements which, while able to convice someone of their truth, didn't leak information
about how they were obtained.
Zero-Knowledge systems proves particularly useful in \emph{ARgument of Knowledge} scenarios
(ZK-ARK): the prover \(P\) wants to convince the verifier \(V\) that he
knows a solution to some problem, assuming there is one, without revealing it.
It must be noted though that known ZK-ARK systems do not guarantee the formal soundness of
the proof: there is a small probability that, given some false statement \(\sigma \) and an
(invalid) proof \(\pi \), then \(V(\pi) = \top \), so it is important to keep this probability
small (say, \(2^{-128}\)).
\begin{definition}[ZK-SNARK]
	Given a prover \(P\), a verifier \(V\), a statement \(\sigma \), and a proof \(\pi \), a 
  Zero-Knowledge Non-interactive ARgument of Knowledge (ZK-SNARK) system is any ZK-ARK system 
  which is:
	\begin{itemize}
		\item \textbf{Succint}: \(SPACE(\pi) = \SmallO(\log(\sigma))\).
		\item \textbf{Non-interactive}: The only communication required by the system is the exchange
		      of \(\sigma \) and \(\pi \).
	\end{itemize}
\end{definition}

\noindent Succintness is an important property in many scenarios, like blockchains, since we 
cannot afford to use too much resources to transmit and store the proofs, and non-interactivity of 
the process allows for efficient verification when multiple parties are involved.

One of the most important applications of ZK-SNARK systems is in \emph{provable computation},
where the prover wants to convince the verifier that he correctly performed some computation
(e.g.\ a cryptocurrency transaction).
\subsection{The Pinocchio Protocol}
A very famous ZK-SNARK system for verifiable computation is the \emph{Pinocchio} protocol,
which was the first one efficient enough to be practical.
Pinocchio uses a lot of mathematical machinery, while we tried to introduce most of the required
background in Section~\ref{sec:preliminaries}, it will still be not trivial to understand fully
\emph{how}, and, more importantly, \emph{why}, it actually works.
We will not go into all of the deep details of the protocol, especially for what concernes the
last stages, but we will still try to explain it in all its parts.

First of all, Pinocchio does not allow the encoding of arbitrary languages, i.e.\ it is not Turing
complete, but we are restricted to arithmetic circuits over some big prime field \(\mathbb{F}_p\)
(tipically, \(p \approx 2^{256}\)).
The main limitation arising from this restriction is that we can only express bounded computation
(i.e.\ no loops whose length depends on some variable condition).
This issue can be mitigated by writing a \emph{circuit synthesizer} in a Turing complete language
which is able to build parametrized arithmetic circuits `on the fly'.
Given an arithmetic circuit \(\phi \) over a prime field \(\mathbb{F}_p\), Pinocchio works as
follows:
\begin{enumerate}
	\item Fix some generator \(g \in \mathbb{F}_p\) such that
	      \(\mathbb{G} = \left\langle{g}\right\rangle \) and an order-preserving non-trivial bilinear
	      map \(B\colon \mathbb{G} \times \mathbb{G} \mapsto \mathbb{G}_{\transpose}\).
	\item Build the R1CS \(\mathcal{C}\) associated with \(\phi \).
	\item Build the QAP \(\mathcal{Q}\) associated with \(\mathcal{C}\).
	\item A trusted third party \(\mathcal{T}\) generates a set of random elements
	      \(R \in \mathbb{F}_p\) which, together with \(g\), are used to build a \emph{prover key}
	      \(K_{\mathcal{P}} \in \mathbb{G}\) of size \(\Theta\left(\left|\phi\right|\right)\) and a
	      \emph{verifier key} \(K_{\mathcal{V}} \in \mathbb{G}\) of size
	      \(\Theta\left(\left|\phi_{IO}\right|\right)\).
	      The random data is deleted immediately after use (\emph{toxic waste}).
	\item The prover know executes \(\phi \), computes all the intermediate values, and uses them to
	      solve \(\mathcal{C}\) and \(\mathcal{Q}\), finding a solution \(\left(p, h\right)\) for
	      \(\mathcal{Q}\).
	\item The prover chooses some value \(x \in \mathbb{F}_p\) with which he computes
	      \(p\left(x\right)\) and \(h\left(x\right)\), and uses \(K_{\mathcal{P}}\) to encrypt them,
	      producing a proof of the kind \(k^{p(x)} =
	      k^{t\left(x\right)h\left(x\right)}\), which has size \(\Theta\left(1\right)\), and is sent
	      to the verifier.
	\item The verifier uses the key \(K_{\mathcal{V}}\), exploiting the bilinear map \(B\), to
	      verify that \(k^{p\left(x\right)} = k^{t\left(x\right)h\left(x\right)}\), which implies
	      that \(p\left(x\right) = t\left(x\right)h\left(x\right)\), which implies with
	      high probability that \(p = th\) which, as we know, implies that \(\phi \) was correctly
	      executed.
\end{enumerate}

\noindent Note how all the work that has to be done from steps 1 to 4 depends only on the
circuit, not on some particular execution, so everything that has been computed in those steps
can be saved and reused later.

\subsection{CHFs for ZK-SNARK systems}
We mentioned that Pinocchio, like other ZK-SNARK systems, works on big prime fields.
The most famous CHFs, like MD5 or the SHA families, make extensive use of bitwise operations,
basically working on some field of the type \(\mathbb{F}_{2^n}\).
When translating these hash functions as arithmetic cirtuits over a prime field \(\mathbb{F}_p\),
we incur in a huge space and time overhead.
\begin{example}
	Consider some \(S, T \in {\left\{0, 1\right\}}^n\) for some \(n \in \mathbb{N}\), and assume we
	want to compute \(S \mathbin{\textsc{xor}} T\).
	If we work over \(\mathbb{F}_{2^n}\) the resulting arithmetic circuit would simply be
	\(x \oplus y\), with \(x = \sum_{i}{S_{i}2^{i-1}}\) and \(y = \sum_{i}{T_{i}2^{i-1}}\).

	On the other hand, to express the same operation over \(\mathbb{F}_p\) we consider \(S\) and
	\(T\) as vectors
	\(\bm{x}, \bm{y} \in \mathbb{F}^n \mid \forall i\colon \bm{x}_i = S_i \wedge \bm{y}_i = T_i\).
	For a single-bit \textsc{xor} operation, we use the circuit
	\(x \oplus y - 2\left(x \otimes y\right) \); to represent an n-bit \textsc{xor}, we concatenate
	\(n\) such circuits.
	What required a single addition in \(\mathbb{F}_{2^n}\) requires \(3n\) additions and \(n\)
	multiplications in \(\mathbb{F}_p\)!

	Furthermore, since \(\bm{x}\) and \(\bm{y}\) are defined over \(\mathbb{F}_p^n\), we must also
	guarantee that their elements are either \(0\) or \(1\), by explicitly adding the contraint
	\(\left(x\right) \otimes \left(x - 1\right) = 0\) for every \(x \in \bm{x}, \bm{y}\).
\end{example}

For this reason, we want CHFs which use native operations over \(\mathbb{F}_p\):
while they are tipically much slower if implemented, say, in x86 assembly, they become
immensely faster when implemented in a ZK-SNARK framework.

\begin{definition}[MiMC primitives]
	Given a finite field \(\mathbb{F}_p\),
	\(r = \left\lceil{\frac{\log_2\left(p\right)}{\log_2\left(3\right)}}\right\rceil \), some
	\(\bm{c} \in \mathbb{F}_p^r\) and the functions:
	\[\forall i < r\colon f_i\left(x,k\right)\colon \mathbb{F}_p \times \mathbb{F}_p \mapsto
		\mathbb{F}_p = x^3 \oplus k \oplus \bm{c}_i\]
	the \emph{MiMC keyed permutation} is defined as:
	\[E\left(x, k\right)\colon \mathbb{F}_p \times \mathbb{F}_p \mapsto \mathbb{F}_p =
		\left(f_{r} \circ \dots \circ f_1\right)\left(x, k\right) \oplus k\]
	By fixing \(k = 0\), we obtain the \emph{MiMC permutation} \(P\left(x\right)\).
	By applying the Davies-Meyer construction to the MiMC permutation, we obtain the
	\emph{MiMC compression function} \(C\left(x\right)\).
	Finally, by applying the Merkle-Damg\r{a}rd construction to the MiMC compression function, we
	obtain the \emph{MiMC hash function} \(H\left(x\right)\).
\end{definition}

