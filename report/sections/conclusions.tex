\section{Conclusions and future directions}\label{sec:conclusions}
In this report we studied the ZK-SNARK Pinocchio protocol, implemented the `Pinocchio-friendly'
hash function MiMC, together with the compact ABR mode of hash, using the \texttt{libsnark} library,
and we tested their performance in the path verification problem comparing them to the traditional
SHA over Merkle Trees combination.

In the last years there has been a number of proposals for efficient primitives in Zero-Knowledge
contexts, which are of particular interest for blockchain and cryptocurrency technologies.
While MiMC is certainly extremely more efficient than SHA in a ZK setting, it seems like there is
still room left for improvement, both in the design of ZK-SNARK systems and in cryptographic
functions.
